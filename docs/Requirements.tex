%% Use Cases Template File
%% Created by Tom Desair (http://www.tomdesair.com)
%% Downloadable at: http://www.tomdesair.com/downloads/use-case-latex-template.zip
%% Date Modified: 03/04/2012
%
% This work may be distributed and/or modified under the
% conditions of the LaTeX Project Public License, either version 1.3
% of this license or (at your option) any later version.
% The latest version of this license is in
%   http://www.latex-project.org/lppl.txt
% and version 1.3 or later is part of all distributions of LaTeX
% version 2005/12/01 or later.

\documentclass[a4paper, 10pt, oneside, draft]{article}

%include the usecases package
\usepackage{usecases}
\begin{document}

%Sometimes it is a good idea to put domain objects in \texttt{}
%The template and the descriptions are based on the book Applying UML and Patterns: 
%An Introduction to Object-Oriented Analysis and Design and Iterative Development
%(3rd Edition) by Craig Larman.
\begin{usecase}

\addtitle{Introduction}{} 



%Main Success Scenario: A typical, unconditional happy path scenario of success.
\addscenario{}{\indent Edith is educational software designed to introduce new computer science students to programming.  New students of any age can learn basic object-oriented programming concepts
    prior to learning a particular programming language in detail (syntax, data types and structures, etc.).  Edith provides a graphical user interface through which
	users can create a "script" that animates a 2-D character, or sprite, provided by the program.  The user creates the script by piecing together provided blocks of 
	program structure in a functional way.  Users may create games and share their creations with others. 
	Similar programs (e.g., Alice) have been shown to increase the interest level and retention of students taking their first programming classes.	
	Edith will be written primarily in JavaScript and will be available over the internet.  After initial development, the software will be open-source to allow 
	further development and additional educational opportunities. 
    \newline

\indent The Visual Editor (VE) module will provide a visual programming language editor (in essence, an integrated development environment(IDE)).  The VE will allow users
	to arrange programming elements to create an animation sequence; programming elements will include methods, variables, control statements, etc. 
	The VE will, while enforcing syntax rules, convert the program into JavaScript, which will be used by other modules.
    \newline

\indent The VE will directly interact with two other Edith modules, Object Creator and Story Creator.  It will provide the Object Creator an interface for defining
	objects, and the Story Creator an interface for defining animations.  The VE will be able to run its own disply view and integrate into the views of 
	other Edith modules.
    \newline

\indent The user will be able to export his program in JavaScript/JavaScript Object Notation (JSON) format. JSON is an efficient data interchange text format that is language independent.
   
}




\end{usecase}


\begin{usecase}

\addtitle{Use Case 1}{Edit variable} 

%Primary Actor: Calls on the system to deliver its services.
\addfield{Primary Actor:}{End-User}



%Preconditions: What must be true on start and worth telling the reader?
\addfield{Preconditions:}{Add method which does have a variable parameter.}
%when multiple
%\additemizedfield{Preconditions:}{} 

%Postconditions: What must be true on successful completion and worth telling the reader
\addfield{Postconditions:}{Variable is defined.}
%when multiple
%\additemizedfield{Preconditions:}{}

%Main Success Scenario: A typical, unconditional happy path scenario of success.
\addscenario{Main Success Scenario:}{
    \item User selects variable
    \item User changes the parameter 
    \item Check type
    \item If it is a compatible type allow input and save parameter
}

%Extensions: Alternate scenarios of success or failure.
\addscenario{Extensions:}{
	\item[2.a] Invalid type data:
		\begin{enumerate}
		\item[1.]  Incorrect input
		\item[2.] User returns to step 1 or exits
		\end{enumerate}

}

%Non-Functional Requirements Needed: Related Non-Functional Requirements Needed.
\additemizedfield{Non-Functional Requirements Needed:}{
	\item Learning Experience.
	\item Usability.
}


%Miscellaneous: Such as open issues/questions
%\addfield{Open Issues:}{}

\end{usecase}

\begin{usecase}

\addtitle{Use Case 2}{Dragging and dropping functionality for methods} 

%Primary Actor: Calls on the system to deliver its services.
\addfield{Primary Actor:}{End-User}



%Preconditions: What must be true on start and worth telling the reader?
\addfield{Preconditions:}{There is a method for which it is possible to be selected.}
%when multiple
%\additemizedfield{Preconditions:}{} 

%Postconditions: What must be true on successful completion and worth telling the reader
\addfield{Postconditions:}{Method is now ready to be use.}
%when multiple
%\additemizedfield{Preconditions:}{}

%Main Success Scenario: A typical, unconditional happy path scenario of success.
\addscenario{Main Success Scenario:}{
    \item User creates new variable and selects it
    \item They can now drag and drop the variable in the boundaries provided


}

%Extensions: Alternate scenarios of success or failure.
\addscenario{Extensions:}{
	\item[2.a] Invalid drop location:
		\begin{enumerate}
		\item[1.] The user attempts to drag the variable outside of     acceptable boundaries, the variable will no be "locked" inside of boundary.

		\end{enumerate}

}

%Non-Functional Requirements Needed: Related Non-Functional Requirements Needed.
\additemizedfield{Non-Functional Requirements Needed:}{
	\item Learning Experience.
	\item Usability.
}


%Miscellaneous: Such as open issues/questions
%\addfield{Open Issues:}{}

\end{usecase}

\begin{usecase}

\addtitle{Use Case 3}{Instantiating  a Conditional Statement } 

%Primary Actor: Calls on the system to deliver its services.
\addfield{Primary Actor:}{End-User}




%Postconditions: What must be true on successful completion and worth telling the reader
\addfield{Postconditions:}{A conditional is instantiated}
%when multiple
%\additemizedfield{Preconditions:}{}

%Main Success Scenario: A typical, unconditional happy path scenario of success.
\addscenario{Main Success Scenario:}{
    \item A user drags and drops a conditional statement
    \item they change the parameter (e.g. if this) 
    \item the inside of the conditional is then draged and dropped

}


%Non-Functional Requirements Needed: Related Non-Functional Requirements Needed.
\additemizedfield{Non-Functional Requirements Needed:}{
	\item Learning Experience.
	\item Usability.
}


%Miscellaneous: Such as open issues/questions
%\addfield{Open Issues:}{}

\end{usecase}

\begin{usecase}

\addtitle{Use Case 4}{Instantiating a Boolean Operator} 

%Primary Actor: Calls on the system to deliver its services.
\addfield{Primary Actor:}{End-User}


%Postconditions: What must be true on successful completion and worth telling the reader
\addfield{Postconditions:}{A boolean operator is instantiated}
%when multiple
%\additemizedfield{Preconditions:}{}

%Main Success Scenario: A typical, unconditional happy path scenario of success.
\addscenario{Main Success Scenario:}{
    \item A user drags and drops an operation (e.g. and, or, not)
    \item the user sets the two variables or expressions
}



%Non-Functional Requirements Needed: Related Non-Functional Requirements Needed.
\additemizedfield{Non-Functional Requirements Needed:}{
	\item Learning Experience.
	\item Usability.
}


%Miscellaneous: Such as open issues/questions
%\addfield{Open Issues:}{}

\end{usecase}


\begin{usecase}

\addtitle{Use Case 5}{ Connecting actions} 

%Primary Actor: Calls on the system to deliver its services.
\addfield{Primary Actor:}{End-User}



%Preconditions: What must be true on start and worth telling the reader?
\addfield{Preconditions:}{There are two or more actions on the development board. }
%when multiple
%\additemizedfield{Preconditions:}{} 

%Postconditions: What must be true on successful completion and worth telling the reader
\addfield{Postconditions:}{The methods are connected.}
%when multiple
%\additemizedfield{Preconditions:}{}

%Main Success Scenario: A typical, unconditional happy path scenario of success.
\addscenario{Main Success Scenario:}{
    \item The method is draged and droped by the user  above or below the action they want to connect to
    \item The user releases the method 
}



%Non-Functional Requirements Needed: Related Non-Functional Requirements Needed.
\additemizedfield{Non-Functional Requirements Needed:}{
	\item Learning Experience.
	\item Usability.
}


%Miscellaneous: Such as open issues/questions
%\addfield{Open Issues:}{}

\end{usecase}

\begin{usecase}

\addtitle{Use Case 6}{Save a Program} 

%Primary Actor: Calls on the system to deliver its services.
\addfield{Primary Actor:}{End-User}



%Preconditions: What must be true on start and worth telling the reader?
\addfield{Preconditions:}{Add method which does have a variable parameter.}
%when multiple
%\additemizedfield{Preconditions:}{} 

%Postconditions: What must be true on successful completion and worth telling the reader
\addfield{Postconditions:}{Variable is defined.}
%when multiple
%\additemizedfield{Preconditions:}{}

%Main Success Scenario: A typical, unconditional happy path scenario of success.
\addscenario{Main Success Scenario:}{
    \item The user selects save this adds the current development board to a list that the user can access
   }

%Extensions: Alternate scenarios of success or failure.
\addscenario{Extensions:}{
	\item[2.a] Unnamed program:
		\begin{enumerate}
		\item[1.]  The user will name the program
		\end{enumerate}

}

%Non-Functional Requirements Needed: Related Non-Functional Requirements Needed.
\additemizedfield{Non-Functional Requirements Needed:}{

	\item Usability.
}


%Miscellaneous: Such as open issues/questions
%\addfield{Open Issues:}{}

\end{usecase}

\begin{usecase}

\addtitle{Use Case 7}{Delete a method} 

%Primary Actor: Calls on the system to deliver its services.
\addfield{Primary Actor:}{End-User}



%Preconditions: What must be true on start and worth telling the reader?
\addfield{Preconditions:}{There are methods on the development board}
%when multiple
%\additemizedfield{Preconditions:}{} 

%Postconditions: What must be true on successful completion and worth telling the reader
\addfield{Postconditions:}{Selected methods are deleted}
%when multiple
%\additemizedfield{Preconditions:}{}

%Main Success Scenario: A typical, unconditional happy path scenario of success.
\addscenario{Main Success Scenario:}{
    \item The user selects a method or group of methods
    \item The user selects to delete the selected items
}

%Extensions: Alternate scenarios of success or failure.
\addscenario{Extensions:}{
	\item[2.a] Invalid type data:
		\begin{enumerate}
		\item[1.]  Incorrect input
		\item[2.] User returns to step 1 or exits
		\end{enumerate}

}

%Non-Functional Requirements Needed: Related Non-Functional Requirements Needed.
\additemizedfield{Non-Functional Requirements Needed:}{
	\item Learning Experience.
	\item Usability.
}


%Miscellaneous: Such as open issues/questions
%\addfield{Open Issues:}{}

\end{usecase}



%Sometimes it is a good idea to put domain objects in \texttt{}
%The template and the descriptions are based on the book Applying UML and Patterns: 
%An Introduction to Object-Oriented Analysis and Design and Iterative Development
%(3rd Edition) by Craig Larman.
\begin{usecase}

\addtitle{Use Case 8}{Run the Program (play)} 

%Primary Actor: Calls on the system to deliver its services.
\addfield{Primary Actor:}{End-User}



%Preconditions: What must be true on start and worth telling the reader?
\addfield{Preconditions:}{Methods have been added to the development board.}
%when multiple
%\additemizedfield{Preconditions:}{} 

%Postconditions: What must be true on successful completion and worth telling the reader
\addfield{Postconditions:}{The program has been run and the state is maintained. }
%when multiple
%\additemizedfield{Preconditions:}{}

%Main Success Scenario: A typical, unconditional happy path scenario of success.
\addscenario{Main Success Scenario:}{
    \item user selects the play button
    \item the program is compiled and if there are no errors the program is run. 
}

%Extensions: Alternate scenarios of success or failure.
\addscenario{Extensions:}{
	\item[2.a] Compile time errors:
		\begin{enumerate}
		\item[1.] Does not run the program
        \item[2.] Highlights error for user
		\end{enumerate}

}

%Non-Functional Requirements Needed: Related Non-Functional Requirements Needed.
\additemizedfield{Non-Functional Requirements Needed:}{
	\item Learning Experience.
	\item Usability.
}


%Miscellaneous: Such as open issues/questions
%\addfield{Open Issues:}{}

\end{usecase}
\begin{usecase}

\addtitle{Use Case 9}{Pause the Program} 

%Primary Actor: Calls on the system to deliver its services.
\addfield{Primary Actor:}{End-User}



%Preconditions: What must be true on start and worth telling the reader?
\addfield{Preconditions:}{The program is running.}
%when multiple
%\additemizedfield{Preconditions:}{} 

%Postconditions: What must be true on successful completion and worth telling the reader
\addfield{Postconditions:}{The state of the program when it was paused is maintained.}
%when multiple
%\additemizedfield{Preconditions:}{}

%Main Success Scenario: A typical, unconditional happy path scenario of success.
\addscenario{Main Success Scenario:}{
    \item the user pauses the program
    \item the program stops  and maintains the current state
}



%Non-Functional Requirements Needed: Related Non-Functional Requirements Needed.
\additemizedfield{Non-Functional Requirements Needed:}{
	\item Learning Experience.
	\item Usability.
}


%Miscellaneous: Such as open issues/questions
%\addfield{Open Issues:}{}

\end{usecase}

\begin{usecase}

\addtitle{Use Case 10}{Instantiating a Loop} 

%Primary Actor: Calls on the system to deliver its services.
\addfield{Primary Actor:}{End-User}



%Preconditions: What must be true on start and worth telling the reader?
\addfield{Preconditions:}{Add method which does have a variable parameter.}
%when multiple
%\additemizedfield{Preconditions:}{} 

%Postconditions: What must be true on successful completion and worth telling the reader
\addfield{Postconditions:}{A loop is instantiated}
%when multiple
%\additemizedfield{Preconditions:}{}

%Main Success Scenario: A typical, unconditional happy path scenario of success.
\addscenario{Main Success Scenario:}{
    \item The user drags and drop a loop to the development board
    \item The user then inputs the conditionals for the loop and its exit conditions. 
}

%Extensions: Alternate scenarios of success or failure.
\addscenario{Extensions:}{
	\item[2.a] Invalid input:
		\begin{enumerate}
		\item[1.]  User returns to step 1 or exits
    	\end{enumerate}

}

%Non-Functional Requirements Needed: Related Non-Functional Requirements Needed.
\additemizedfield{Non-Functional Requirements Needed:}{
	\item Learning Experience.
	\item Usability.
}


%Miscellaneous: Such as open issues/questions
%\addfield{Open Issues:}{}

\end{usecase}

\begin{usecase}

\addtitle{Use Case 11}{Creating a new Action} 

%Primary Actor: Calls on the system to deliver its services.
\addfield{Primary Actor:}{End-User}



%Preconditions: What must be true on start and worth telling the reader?
\addfield{Preconditions:}{None.}
%when multiple
%\additemizedfield{Preconditions:}{} 

%Postconditions: What must be true on successful completion and worth telling the reader
\addfield{Postconditions:}{If first method, the box is made, if attaching to another method, the boxes are connected}
%when multiple
%\additemizedfield{Preconditions:}{}

%Main Success Scenario: A typical, unconditional happy path scenario of success.
\addscenario{Main Success Scenario:}{
    \item The user selects the new method button
    \item Drags the method into the workbox 
    \item then the method prompts the user to give arguments for that specific action 
    \item If arguments are compatible, the method is created

}

%Extensions: Alternate scenarios of success or failure.
\addscenario{Extensions:}{
	\item[2.a] Invalid input:
		\begin{enumerate}
		\item[1.]  User prompted to re-enter arguments
    	\end{enumerate}

}

%Non-Functional Requirements Needed: Related Non-Functional Requirements Needed.
\additemizedfield{Non-Functional Requirements Needed:}{
	\item Learning Experience.
	\item Usability.
}


%Miscellaneous: Such as open issues/questions
%\addfield{Open Issues:}{}

\end{usecase}

\begin{usecase}

\addtitle{Use Case 12}{Connecting actions} 

%Primary Actor: Calls on the system to deliver its services.
\addfield{Primary Actor:}{End-User}



%Preconditions: What must be true on start and worth telling the reader?
\addfield{Preconditions:}{There are two actions on the development board.}
%when multiple
%\additemizedfield{Preconditions:}{} 

%Postconditions: What must be true on successful completion and worth telling the reader
\addfield{Postconditions:}{The methods are connected}
%when multiple
%\additemizedfield{Preconditions:}{}

%Main Success Scenario: A typical, unconditional happy path scenario of success.
\addscenario{Main Success Scenario:}{
    \item The action is selected by the user
    \item The action is dragged above or below the action they want to connect to
    \item Release to connect

}

%Extensions: Alternate scenarios of success or failure.
\addscenario{Extensions:}{
	\item[2.a] Invalid Action Placement:
		\begin{enumerate}
		\item[1.]  If the user misses the action while dropping another one on it or the user drops the action in a place out of the frames range, the action will be returned to where it got picked up from. 
    	\end{enumerate}

}

%Non-Functional Requirements Needed: Related Non-Functional Requirements Needed.
\additemizedfield{Non-Functional Requirements Needed:}{
	\item Learning Experience.
	\item Usability.
}


%Miscellaneous: Such as open issues/questions
%\addfield{Open Issues:}{}

\end{usecase}


\begin{usecase}

\addtitle{Use Case 13}{Save a collection of methods} 

%Primary Actor: Calls on the system to deliver its services.
\addfield{Primary Actor:}{End-User}



%Preconditions: What must be true on start and worth telling the reader?
\addfield{Preconditions:}{There is a method to be saved on the board.}
%when multiple
%\additemizedfield{Preconditions:}{} 

%Postconditions: What must be true on successful completion and worth telling the reader
\addfield{Postconditions:}{The method is saved}
%when multiple
%\additemizedfield{Preconditions:}{}

%Main Success Scenario: A typical, unconditional happy path scenario of success.
\addscenario{Main Success Scenario:}{
    \item The user selects the method or the set of actions that they want to save. The method gets highlighted
    \item The user clicks the save button
    \item Method is added to list of actions that the User can access. 
    \item User is prompted to name method
    \item Method is created after it is named
    

}

%Extensions: Alternate scenarios of success or failure.
\addscenario{Extensions:}{
	\item[2.a] Invalid Action Sequence:
		\begin{enumerate}
		\item[1.]  If the user selects the set of actions, and then clicks somewhere other than the save button, the set of actions becomes deselected
    	\end{enumerate}

}

%Non-Functional Requirements Needed: Related Non-Functional Requirements Needed.
\additemizedfield{Non-Functional Requirements Needed:}{
	\item Learning Experience.
	\item Usability.
}


%Miscellaneous: Such as open issues/questions
%\addfield{Open Issues:}{}

\end{usecase}

\begin{usecase}

\addtitle{Use Case 14}{Delete a Method} 

%Primary Actor: Calls on the system to deliver its services.
\addfield{Primary Actor:}{End-User}



%Preconditions: What must be true on start and worth telling the reader?
\addfield{Preconditions:}{There is a method to be deleted on the board.}
%when multiple
%\additemizedfield{Preconditions:}{} 

%Postconditions: What must be true on successful completion and worth telling the reader
\addfield{Postconditions:}{The method is dleted}
%when multiple
%\additemizedfield{Preconditions:}{}

%Main Success Scenario: A typical, unconditional happy path scenario of success.
\addscenario{Main Success Scenario:}{
    \item The user selects the method or the set of actions that they want to save. The method gets highlighted
    \item The user clicks the delete button
    \item Method is deleted
    

}

%Extensions: Alternate scenarios of success or failure.
\addscenario{Extensions:}{
	\item[2.a] Invalid Action Sequence:
		\begin{enumerate}
		\item[1.]  If the user selects the set of actions, and then clicks somewhere other than the delete button, the set of actions becomes deselected
    	\end{enumerate}

}

%Non-Functional Requirements Needed: Related Non-Functional Requirements Needed.
\additemizedfield{Non-Functional Requirements Needed:}{
	\item Learning Experience.
	\item Usability.
}


%Miscellaneous: Such as open issues/questions
%\addfield{Open Issues:}{}

\end{usecase}




\end{document}