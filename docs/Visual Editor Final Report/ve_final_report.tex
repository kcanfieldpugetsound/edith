\documentclass[a4paper]{article}

\usepackage[english]{babel}
\usepackage[utf8x]{inputenc}
\usepackage{amsmath}
\usepackage{graphicx}
\usepackage[colorinlistoftodos]{todonotes}

\title{CS240: Individual Report}
\author{Vikram S. Nilakantan}

\begin{document}

%-------------------------------------------------------------------------------------------------------------
% TITLE PAGE
%-------------------------------------------------------------------------------------------------------------

\begin{titlepage}
        \vspace*{\fill} %leave out given verticle space in a document
        \begin{center}
                {\Huge Visual Editor Final Report}\\ [0.5cm]        
                
                %make title huge and have .5cm space in between
                {\Large Graham Baker, Walker Bohannen, Jessica Lefton, \\Steve Marx, Vikram Nilakantan, Eli Spiegel}\\[0.4cm]
                \today %put the date that the data is compiled
        \end{center}
        \vspace*{\fill}
\end{titlepage}

\section{Project Summary}

\section{Development Procedures}

\section{Requirements Evaluation}

\section{System Design and Architecture}

\section{Reflection: Vikram Nilakantan}
\subsection{Challenges}
\subsubsection{Technical Challenges}

The biggest technical challenge that I faced while on the Visual Editor team was my personal inexperience with the HTML5 Canvas or any of the graphical JavaScript libraries. After learning that the visual editor was going to be based in an HTML5 Canvas, I started to go through the process of figuring out how it worked and how to use it. I was told that the Canvas was very similar to doing graphics in Java, but since I had little to no experience with Java graphics, that advice did not help me as much. Two things mainly helped me overcome this challenge. The first was my prior experience using JavaScript and HTML. Even though the Canvas was unfamiliar, the language used to control it was very familiar and I was able to use my prior knowlege to accelerate the learning process. The second learning tool was simply trial-and-error. No one on the visual editor team had ever used the HTML5 Canvas or KineticJS before so what really helped us was several test documents trying to figure out how object were created and how we could interact with them. 

\subsubsection{Organizational Challenges}

Unlike other groups working on the Edith project, our group had six members, opposed to five. Since none of us had any familiarity with the technologies we were working with, our organizational challenges revolved around slow progress. Another challenge that surrounded the visual editor is that it was very difficult to seperate work out until we had a solid foundation of what we were doing with Canvas and KineticJS and how we were going to execute such actions. Until we got to a point where we had a plan of action in which we knew how things were going to be done, we were unable to work seperately and the only time progress was made was when all of us could meet up and surround one computer while we tried things out. Fortunately, we soon figured out what we were trying to do and were clearly able to divide tasks among group members.

\subsection{Applied Software Engineering}

Many aspects of the visual editor were designed and implemented according to software engineering techniques, however, I thought that the biggest lesson from software engineering we used was on creating the user interface. We estimated that the visual editor component of Edith is where the typical end-user would spend the majority of their time. Because of this, we wanted to spend some time on designing the user interface and become aware of exactly what actions users would be able to perform and make sure that certain elements of the editor are presented in a consistent manner.

\subsection{Different Approach}

One of the largest over-arching challenges that the visual editor team faced was that we were unsure of our requirements for the project and I felt like each of us had 'an idea' of what we were doing, but no one's ideas actually matched up with anyone else's. This posed a big problem at the beginning of this project and until we were sure of what we were doing, our progress stalled and at some points, came to a halt. If I were to do this project again, I would make sure that I was completely aware of the project/section requirements. If I were given more time to continue to work on this project, I would want to improve some of the graphical elements and transitions on the visual editor. The boxes and 'method containers' used currently are functional, but I feel like with more time, I could make those boxes more visually appealing. Another feature I would implement is ease-of-use for the end-user. That is, making sure it is clear how to operate the visual editor and making sure the user is not confused when they are trying to operate the application.


\section{Reflection: Graham Baker}
\subsection{Challenges}
Some of the main challenges I experienced during development were learning a new language, dealing with GitHub, and trying to find dividing lines between our group and others. I really enjoyed learning to use javascript. While it was pretty difficult at the beginning to work out how to access certain attributes, using global variables and other general points of confusion transitioning from traditional object oriented programming languages, I felt just pounding through trying to get my code to work was extremely beneficial and valuable. GitHub was also a source of confusion and frustration early on. I used the terminal for the first half of the semester and struggled to accomplish what I wanted to do. After I switched to the GitHub application, things got a lot easier. I am glad that I was introduced and got experience working with this important software development tool. Finally, the most difficult challenge this semester was settling on what each group was supposed to do. There were a few times where we were not really sure what we were supposed to be doing because our requirements overlapped with other groups. It was a good experience for us to have to grapple with the miscommunication and complexity that is so often the downfall of many projects. I learned that it is vital to get the requirements and direction of the project as early as possible to make life easier as the project goes on. 


\subsection{Applied Software Engineering}
The technique that I most frequently used over the course of the semester was the idea of iterative development. Since our requirements and goals were constantly shifting, we had to adapt every time we got a new direction. It was also nice to see that is consistent with how most software engineering firms operate.   

\subsection{Different Approach}

If we started over, I think we would probably seek out another graphics package to use for our user interface. While kinetic.js offered quite a bit of useful tools, I think one that supported text input would have probably suited our needs better. We landed on kinetic.js after trying to get things to work with jquery ui as well as other graphics packages. As we approached the first implementation, we just had to go with kinetic.js. If we wanted to continue development we could find a better package that would suit our needs more appropriately. Now that we are at a decent position, I think there is a lot that could be done. It would be really great to add method box encapsulation like we wanted to do. Now that things are up and running, it is easy to see things that could be improved on and how to make the software easier to use for the user. 

\section{Reflection: Eli Spiegel}
\subsubsection{Technical and organizational Challenges}
The project started trying to use the waterfall method, however as it progressed the lack of direction made the waterfall method challenging. This led to our team and a few others moving to a more agile approach. Agile helped us get a product that worked in a shorter amount of time as we were able to progress and find different problems instead of spending all of our time planning and then quickly finding new issues. Other issues included communication between people with diverse schedules, and working with tools and languages I was unfamiliar with, mostly JavaScript and GitHub.

\subsection{Starting this Project Over}
I think writing requirements as a class with Joel as the “client” could have made for a better use of the waterfall method as everyone would have a better understanding of the overall project, which would help with moving forward on the individual modules and solve division of labor/dependency issues. Another issue the project had was a lack of experience with the languages/tools groups would be using took the focus off of the software engineering problems and made them more implementation problems. I believe that a stronger focus on learning the tools and languages at the start of a project will make it clear what problems may arise and possible ways to solve them. 

\subsection{Learning}
In the end some of the more valuable things I learned were how to write code as a group, and how to use tools such as GitHub. I also learned that knowledge is the most important tool in software development.  If a group comes in knowledgeable about the tools and about what the goal is that they will build even complex projects in fairly short about of time.

\end{document}