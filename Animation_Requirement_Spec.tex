%http://cs.pugetsound.edu/~jross/courses/cs240/project/requirements/
%Animation Group
\documentclass[12pt]{article}
\usepackage{graphicx}
\begin{document}




% Front Page
\begin{titlepage}
	\begin{center}
	\huge  Edith \\
	\vspace*{\fill}%
 	\huge \textsc{\textbf{Animation Team \\Requirement Specifications} }	
	\bigskip 
	\rule{130mm}{.1pt}
	\textsc{\textbf{September 25 2013 \\ Revised: September 25 2013} \\ }	
	\vspace*{\fill}%
	Eric Lund \\
	Kramer Canfield \\ 
	Zeke Rosenberg \\
	Calder Whiteley \\
	Jon Youmans
	\end{center}
	\end{titlepage}

% Next page

%Executive Summary
\section{\emph{Executive Summary}}
Our Summary will be placed here!Our Summary will be placed here!Our Summary will be placed here!Our Summary will be placed here!Our Summary will be placed here!Our Summary will be placed here!Our Summary will be placed here!Our Summary will be placed here!Our Summary will be placed here!


%Introduction
\section{\emph{Introduction}}%Create a section for the introduction
	\subsection{Edith}
Details hereDetails hereDetails hereDetails hereDetails hereDetails hereDetails hereDetails hereDetails hereDetails hereDetails hereDetails hereDetails hereDetails hereDetails hereDetails hereDetails hereDetails hereDetails here
	\subsection{Module}
	Details hereDetails hereDetails hereDetails hereDetails hereDetails hereDetails hereDetails hereDetails hereDetails hereDetails hereDetails hereDetails hereDetails hereDetails hereDetails hereDetails hereDetails hereDetails hereDetails hereDetails hereDetails hereDetails hereDetails hereDetails hereDetails hereDetails hereDetails hereDetails hereDetails hereDetails hereDetails hereDetails hereDetails hereDetails hereDetails hereDetails hereDetails here
	\subsection{Purpose}
	Details hereDetails hereDetails hereDetails hereDetails hereDetails hereDetails hereDetails hereDetails hereDetails hereDetails hereDetails hereDetails hereDetails hereDetails hereDetails hereDetails hereDetails hereDetails here	Details hereDetails hereDetails hereDetails hereDetails hereDetails hereDetails hereDetails hereDetails hereDetails hereDetails hereDetails hereDetails hereDetails hereDetails hereDetails hereDetails hereDetails hereDetails here


% Use Cases
\section{\emph{Functional Requirements}}
	\subsection{"Use Case 1 NAME"}
\begin{enumerate}
  \item Actor
  \begin{enumerate}
  		\item "Drawer": The system providing us with sprites to animate and sounds to play.
   		 \item "Taker": The system that takes and displays our canvas.
		\item Animation System: The system we are creating that handles back-end animation and final.
  \end{enumerate}
  \item Preconditions/Assumptions
  \begin{enumerate}
   		 \item Preconditions: The user has already defines which sprites will be animated and which sounds will be played.
   		 \item Assumptions: The web browser supports HTML5 and JavaScript.
  \end{enumerate}
  \item Flow of Events
  \begin{enumerate}
   		 \item "Drawer" provides drawing instructions to the Animation System. 
   		 \item The Animation System processes the instructions.
		\item The Animation System creates animations and plays sounds.
		\item The Animation System provides the animations to the "Taker."
  \end{enumerate}
  \item Alternatives
  \begin{enumerate}
    		\item The "Drawer" provides faulty drawing instructions.
    		\item The Animation System attempts and fails to process instructions.
		\item The Animation System passes the error to the "Taker."
  \end{enumerate}
  %\item  Postconditions
  %\begin{enumerate}
  % 		\item 
  %\end{enumerate}
\end{enumerate}

	

\subsection{UML}
\includegraphics[scale=.7]{UML.png}
UML Diagram of Animation System Flow of Events


%Nonfunctional Requirements
\section{\emph{Nonfunctional Requirements}}
L

%Glossary/References
\section{\emph{Glossary/References}}
Finally, be sure to define the specialized terms you use (if any!), and to include citations to any references you make (e.g., if you reference any other systems as comparison points). Always provide proper attribution to other people's work.

\end{document}